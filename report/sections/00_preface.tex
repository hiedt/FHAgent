\begin{acknowledgements}
I would like to express my deepest gratitude to Professor Ali Khosravi. His insightful expertise has provided me with invaluable guidance to conduct this research to the highest academic standards.

I am also thankful to my supervisors at OJ Electronics A/S, particularly Rasmus L. Koldsø and Junior A. Ndifor, for their unwavering support in almost every aspect of this thesis.

Then, a special thanks to Davi Gonçalves Accioli---a friend, a mentor, and above all, a free human-GPT who is full of patience to answer my dumbest questions.
    
Finally, here is the most heartfelt love for my mom, dad, brother, especially my dear girlfriend, Lê Khánh Vi. Your encouragement and mental support have always been an endless source of motivation. Thank you for believing in me. I love you all.
\end{acknowledgements}

\begin{abstract} 
Previous studies have demonstrated the efficacy of reinforcement learning in various control tasks. Still, its application to floor heating has remained largely unexplored due to the system's slow dynamics and strict safety requirements. This research addresses this gap by developing a new model tailored for real-world settings, aiming to answer if it can surpass the performance of the existing method. Our approach includes a thorough background survey, a comprehensive literature review, and careful mathematical simulation, followed by an extensive assessment of five obscure edge cases. The results indicate that our new controller consumes up to 8.62 (\%) less electricity with nearly two relay switches as few per hour, maintains a small tolerance of $\pm 0.2$ (°C), and barely overshoots. These findings suggest that reinforcement learning holds significant potential for enhancing thermal systems, although further attention to safety protocols and intrinsic dynamical changes is necessary for practical integration.
\end{abstract}

%% Uncomment the `iknowhattodo' option to dismiss the instruction in the PDF.
% \begin{publishedcontent}%[iknowwhattodo]
% % List your publications and contributions here.
% \nocite{Cahn:etal:2015,Cahn:etal:2016}
% \end{publishedcontent}

\tableofcontents
\listoffigures
\listoftables
\printnomenclature

\mainmatter